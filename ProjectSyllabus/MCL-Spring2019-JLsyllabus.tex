\documentclass[10pt]{article}

\usepackage{amssymb,amsmath,hyperref,fullpage}
\pagestyle{empty}

\newcommand\R{\mathbf{R}}

\begin{document}

\def\vspacing{\vspace{.6cm}}

\noindent

\begin{center}
\large \textbf{VIPICS}: \textbf{V}isualizing and \textbf{I}nteracting with \textbf{P}aths in \textbf{C}onfiguration \textbf{S}paces
\\[5pt]\normalsize \textbf{Mathematics Computing Laboratory} \\
Spring 2019 project
\end{center}

\vspacing
 
\noindent
\textbf{Project supervisor:} J\=anis Lazovskis, SEO 522 \\
\textbf{Website:} \href{https://github.com/jlazovskis/vipics}{\texttt{github.com/jlazovskis/vipics}} \\
\textbf{Email:}  \href{mailto:jlazov2@uic.edu}{\nolinkurl{jlazov2@uic.edu}}

\vspacing

\noindent
\textbf{Structure.}
\begin{itemize}
\item There will be weekly meetings with assigned readings / problems / coding. 
\item You will recieve a letter grade based on your participation.
\end{itemize}

\vspacing

\noindent
\textbf{Checklist (non-math).} 
\begin{itemize}
\item how to use \texttt{git} and GitHub
\item how to code in C\#
\item how to use Unity with the Oculus headset
\end{itemize}

\vspacing

\noindent
\textbf{Checklist (math).} Undergraduate level understanding of sets and algebraic topology.
\begin{itemize}
\item Edelsbrunner, Harer: Chapter 3
\item Carlsson: Sections 2.1, 2.3
\item Aguilar, Gitler, Prieto: Pages xvii-xx
\item Hatcher: Pages xii, 5-6, 25-27
\end{itemize}

\vspacing
\noindent
\textbf{Goals (non-math).}  Primary, secondary, and tertiary, all within a Unity scene.
\begin{enumerate}
\item[P1.] Create a scene where, with the Oculus controls, the user can
\begin{enumerate}
\item add and delete points in $\R^3$,
\item ajust the real parameter $r \in \R_{\geqslant 0}$ with one of the joysticks.
\end{enumerate}
\item[P2.] Visualize the Vietoris--Rips complex from the points in the scene and the radius, which changes as the user moves the points and adjusts the radius.
\item[P3.] Create a poster describing the semester's work.
\item[S1.] Pair points together to describe straight-line paths in space, and let the user adjust position along the paths with the other joystick. The VR complex is visible and responds to user changes.
\item[T1.] Adjust the visualizations for the \v Cech complex instead. Allow the user to switch between them.
\item[T2.] Visualize the resulting stratified two-dimensional space.
\end{enumerate}

\vspacing

\noindent
\textbf{Goals (math).}
\begin{enumerate}
\item[P1.] Understand, work with, and compute
\begin{enumerate}
\item simplicial homology,
\item the topology of and distances in configuration space.
\end{enumerate}
\item[S1.] Given $P\subseteq \R^N$ of size $n$, describe a formula that gives $r\in \R_{\geqslant 0}$ at which the $(n-1)$-simplex of the \v Cech construction is born.
\item[T1.] Understand, work with, and compute distances between persistence diagrams.
\end{enumerate}

\vspacing

\noindent
\textbf{Sources.} Some, not all.
\begin{itemize}
\item Aguilar, Gitler, Prieto (2002). \textit{Algebraic Topology from a Homotopical Viewpoint.}
\item Carlsson (2009). \textit{Topology and data.}
\item Chan, Carlsson, Rabadan (2013). \textit{Topology of viral evolution.}
\item Edelsbrunner, Harer (2009). \textit{Computational Topology: An Introduction.}
\item Hatcher (2015). \textit{Algebraic Topology.}
\item May (1999). \textit{A Concise Course in Algebraic Topology.}
\item Topaz, Ziegelmeier, Halverson (2015). \textit{Topological Data Analysis of Biological Aggregation Models.}
\end{itemize}

\noindent


\end{document}